\documentclass[12pt, a4paper]{article}


\usepackage{ctex}
\usepackage{amsmath}           % 数学公式支持
\usepackage{geometry}          % 页面布局设置
\usepackage{enumitem}          % 增强列表环境,用于调整间距
\usepackage{xcolor}            % 颜色支持,用于美化链接样式
\usepackage{listings}
\usepackage{chemfig}
\usepackage[version=4,arrows=pgf-filled,
textfontname=sffamily,
mathfontname=mathsf]{mhchem}

\lstset{
    basicstyle=\ttfamily\small,           % 基础样式:等宽字体 + 小号字
    breaklines=true,                      % 自动换行
    keywordstyle=\color{blue}\bfseries,   % 关键字颜色:蓝色粗体
    commentstyle=\color{gray},            % 注释颜色:灰色
    stringstyle=\color{red},              % 字符串颜色:红色
    numbers=left,                         % 行号显示在左侧
    numberstyle=\tiny\color{gray},        % 行号样式
    stepnumber=1,                          % 行号步进
    % frame=single,                          % 代码块加边框
    showspaces=false,                      % 不显示空格标记
    showstringspaces=false,                % 字符串中不显示空格标记
    tabsize=2                              % Tab键宽度
}

% ===== 页面布局 =====
\geometry{left=2.5cm, right=2.5cm, top=3cm, bottom=3cm}


\newcommand{\wikilink}[1]{\textcolor{blue}{\texttt{#1}}}

\setlist{nosep, topsep=0pt}

\title{ 使用Nix 构建 \LaTeX 文档}
\author{Butterblock233}
\date{ \today }
\begin{document}

\maketitle

\section{概述}
Nix是一款基于纯函数式的Nix语言的包管理器。得益于纯函数式的特性,Nix搭建的环境具有极强的可复现性。除了网络错误几乎不会导致构建出错。本文主要介绍了利用Nix来搭建一个可靠的构建环境来编写 \LaTeX 文档。本仓库也能作为现成的模板仓库,帮助用户绕开搭建环境的烦恼,专心创作内容。

\section{构建此文档}
\subsection{安装并配置Nix}
NixOS用户只需要确保自己已经启用了flakes和new-command特性即可。

非NixOS用户首先需要安装好Nix包管理器,并配置启用flakes和new-command新特性。
\subsection{构建}
如果你已经安装好Nix,接下来只需要\verb|nix build|即可执行完整的构建流程,构建产生的PDF文件能够在\verb|result/|里找到。

\section{工具链介绍}
\subsection{Nix}
Nix是基于纯函数式语言Nix的包管理器。Nix和NixOS项目直接起源于其创始人 Eelco Dolstra 的博士论文工作,这奠定了它独一无二的理论基础。在他的论文里提出了如下思想:
\begin{itemize}
\item 纯函数式构建:将软件构建过程建模为一个纯函数。函数的“输入”是所有构建依赖(源码、编译器、库、环境变量等),函数的“输出”是构建结果(软件包)。
\item hash路径存储:nix的软件包全部存储在例如\lstinline|/nix/sha256-python3.11|的路径内,这样nix可以精确检索,同时确保了软件包能够共存。
\end{itemize}
\begin{itemize}
\item Nix:主要的包管理工具,能够创建一个可靠的环境,通过\verb|nix build|一键构建
\item LaTeX Workspace:流行的VSCode插件,提供了构建、实时预览等实用功能
\item Nix-env selector: 在VSCode中使用Nix环境
\item direnv:在终端中自动进入Nix环境,而不需要手动执行\verb|Nix develop|
\end{itemize}


\end{document}