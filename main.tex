\documentclass[12pt, a4paper]{article}


\usepackage{ctex}
\usepackage{amsmath}
\usepackage{geometry}
\usepackage{enumitem}
\usepackage{xcolor}
\usepackage{listings}
\usepackage{chemfig}
\usepackage[version=4,arrows=pgf-filled,
textfontname=sffamily,
mathfontname=mathsf]{mhchem}
\usepackage{minted}
\usepackage{fontspec}
\usepackage{hyperref}

% TBD: 使用JetBrains Mono会导致构建失败,但JetBrains Mono NL可以
\setmonofont{JetBrains Mono NL}
\setCJKmonofont{Source Han Sans SC}
\geometry{left=2.5cm, right=2.5cm, top=3cm, bottom=3cm}
\setlist{nosep, topsep=0pt}

\title{ 使用Nix 构建 \LaTeX 文档}
\author{\href{https://github.com/Butterblock233}{Butterblock233}}
\date{ \today }


\begin{document}

\maketitle

\section{概述}
\LaTeX 是当今使用最广泛的专业排版系统之一,能构建高质量的文档。是论文写作等专业出版的必备工具之一。但由于\LaTeX 自身的诸多问题,导致环境的搭建繁琐且容易出错。

Nix是一款基于纯函数式的Nix语言的包管理器。Nix将整个构建过程定义为一个巨大的纯函数。例如本项目中,\LaTeX 实现、字体等软件包作为构建输入,构建输出为构建\TeX 源文件产生的PDF。相同的构建输入一定能够产生相同的构建输出。

本文主要介绍了利用Nix来搭建一个可靠的构建环境来编写 \LaTeX 文档。本仓库也能作为现成的模板仓库,帮助用户避开搭建环境的烦恼,专心创作内容。

\section{构建此文档}
本项目使用LaTeX编写文档,来证明Nix管理的项目环境的确稳定可靠。
\subsection{安装并配置Nix}
\begin{itemize}
\item{对于NixOS用户}: \\
NixOS自带Nix包管理器,用户只需要确保自己已经启用了flakes和new-command特性。在配置中加入以下选项,
\begin{minted}{nix}
    nix.settings.experimental-features = [
        "nix-command"
        "flakes"
    ];
\end{minted}
重新构建系统即可启用。
\item{对于非NixOS用户}: \\
非NixOS用户首先需要安装好Nix包管理器,并配置启用flakes和new-command新特性。
\end{itemize}

\subsection{验证安装}
\noindent 安装完成后,在项目中执行\verb|nix flake show|命令来验证,通常可以得到以下类似结果:
\begin{minted}[obeytabs]{shell}
    git+file:///home/butter/src/tex
    └───packages
        ├───aarch64-darwin
        │   └───default omitted (use '--all-systems' to show)
        ├───aarch64-linux
        │   └───default omitted (use '--all-systems' to show)
        ├───x86_64-darwin
        │   └───default omitted (use '--all-systems' to show)
        └───x86_64-linux
            └───default: package 'pdf'
\end{minted}
\subsection{构建}
安装好Nix之后,接下来只需要执行\verb|nix build|即可开始构建,产生的PDF文件能够在\verb|result/|里找到。

\section{工具链介绍}
\subsection{Nix}
Nix是基于纯函数式语言Nix的包管理器。Nix和NixOS项目直接起源于其创始人 Eelco Dolstra 的博士论文工作,这奠定了它独一无二的理论基础。在他的论文里提出了如下思想:
\begin{itemize}
\item 纯函数式构建:将软件构建过程建模为一个纯粹的巨大的函数。函数的“输入”是所有构建依赖(源码、编译器、库、环境变量等),函数的“输出”是构建结果(软件包)。
\item hash路径存储:nix的软件包全部存储在例如\lstinline|/nix/sha256-python3.11|的路径内,这样nix可以精确检索,同时确保了软件包能够共存。
\end{itemize}

\bigskip

Nix既可以作为项目的包管理器提供了稳定可复现的环境,也可以直接管理整个NixOS系统的所有配置。Nix利用声明式的方法配置其功能,从而避免了运维人员需要手动输入大量命令。一个简单的nix配置模块可能长这样:
\begin{minted}[]{Nix}
{...}:
{
    # 启用Neovim包
    packages.neovim.enable=true;
}
\end{minted}

\subsection{flakes}
flakes是Nix近年引入的一项新特性。回顾一下前面的Nix配置,不难发现,该模块只声明的构建输出(启用Neovim包),没有声明构建输入(Neovim软件包从哪而来)。这导致结果不完全确定。flakes正是为了解决这一问题而生的。一份简化的 NixOS with flakes 配置如下:
\begin{minted}{nix}
{
  description = "A simple NixOS flake config";

  inputs = {
    # nigpkgs输入源,并锁定了25.11版本
    nixpkgs.url = "github:NixOS/nixpkgs/nixos-25.11";
    # nispkgs-unstable源
    nixpkgs-unstable.url = "github:NixOS/nixpkgs/nixos-unstable";
  };
  outputs =
    {
      self,
      nixpkgs,
      nixpkgs-unstable,
      ...
    }@inputs:
    {
      nixosConfigurations = {
        wsl = nixpkgs.lib.nixosSystem {
          # configuration for WSL distro
          system = "x86_64-linux";
          modules = [ ];
        };
        # configuration for remote machines
        remote = nixpkgs.lib.nixosSystem {
          system = "x86_64-linux";
          specialArgs = { inherit inputs; };
          modules = [ ];

        };
      };
    };
}
\end{minted}
通过以上配置我们可以看出,一个flake的核心是input与output。flake对input的锁定机制确保了构建结果的稳定。

本项目的flake.nix主要负责定义输入源,具体输出过程则定义于default.nix
\subsection{其它工具链}
\begin{itemize}
    \item LaTeX Workspace:流行的 \LaTeX\ for VSCode插件,提供了构建、实时预览等实用功能。笔者尤其喜欢保存时自动构建并刷新预览的功能
    \item Nix-Env Selector:在VSCode中使用Nix环境,以确保LaTeX Workspace能够正确使用Nix提供的环境。\bigbreak
    直接执行\verb|nix build|稳定且可靠,但构建耗时较长。为了确保良好的预览、编辑体验,笔者对LaTeX Workspace的构建流程做了一些额外的修改(具体位于.vscode/settings.json)。因此需要加载Nix环境供LaTeX Workspace使用。\bigbreak
    在安装并启用Nix-Env Selector后,加载由default.nix提供的环境即可。加载flakes.nix提供的环境会报错,具体原因有待排查。
    \item direnv:在终端中自动进入Nix环境,而不需要每次手动执行\verb|Nix develop|
\end{itemize}

\section{字体配置}
\noindent XeLaTeX支持现代的OTF格式。利用以下\href{https://stackoverflow.com/questions/5109550/how-to-get-a-list-of-all-available-ttf-fonts-with-xetex#comment105803157_28143009}{命令}列出可用的字体格式:
\begin{minted}{shell}
fc-list | cut -d\  -f2-99 | cut -d: -f1 | sort -u
\end{minted}
要安装新的字体,只需要在default.nix中修改fonts列表即可\pagebreak
\begin{minted}{nix}
fonts = with pkgs; [
    source-han-serif
    source-han-sans
    jetbrains-mono
    # other fonts
    ...
  ];
\end{minted}

\section{延伸阅读}
本文主要介绍了利用Nix来搭建稳定、可靠的\LaTeX 环境,对于如何学习 \LaTeX 等内容,可以参考以下文章:
\end{document}